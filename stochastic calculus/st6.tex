\documentclass[a4paper, 10pt]{article}    
\usepackage{geometry}       
\geometry{a4paper}
\geometry{margin=1in} 
\usepackage{paralist}
  \let\itemize\compactitem
  \let\enditemize\endcompactitem
  \let\enumerate\compactenum
  \let\endenumerate\endcompactenum
  \let\description\compactdesc
  \let\enddescription\endcompactdesc
  \pltopsep=\medskipamount
  \plitemsep=1pt
  \plparsep=1pt
\usepackage[english]{babel}
\usepackage[utf8]{inputenc}

\usepackage{bbm, bm}
\usepackage{amsmath, amssymb, amsthm, mathrsfs}
\usepackage{booktabs, tikz, array, eurosym}
\usepackage{float}
\renewcommand{\arraystretch}{1.4}
\newcolumntype{L}{>{\arraybackslash}m{10cm}}
\newcommand\indep{\protect\mathpalette{\protect\indeP}{\perp}}
\def\indeP#1#2{\mathrel{\rlap{$#1#2$}\mkern2mu{#1#2}}}

\pagestyle{headings}
\newcommand{\boxwidth}{430pt}

\theoremstyle{definition}
\newtheorem{problem}{Problem}

\newtheoremstyle{hSol}
  {1.0pt}% Space above
  {1.0pt}% Space below
  {}% bodyfont
  {}% indent
  {\bfseries}% thm head font
  {.}% punctuation after thm head
  { }% Space after thm head
  {}% thm head spec

\theoremstyle{hSol}
\newtheorem*{solution}{Solution}


\title{\textbf{Stochastic Calculus Assignment VI}}
\author{Ze Yang~~~~(zey@andrew.cmu.edu)}

\begin{document}
\maketitle

~\\
\textit{Note:} \textit{I will use the following notations throughout this assignment:}
\begin{itemize}
	\item[$\cdot$] $W_t := W(t)$.
	\item[$\cdot$] $\Delta_{s,t} := W(t) - W(s)$.
	\item[$\cdot$] $X\in m \mathcal{F}$: $X$ is $\mathcal{F}$-measurable.
	\item[$\cdot$] $X\perp \mathcal{F}$: Random variable $X$ is independent to sigma algebra $\mathcal{F}$. 
	\item[$\cdot$] $X\stackrel{d}{=} Y$: Random variable $X,Y$ have identical distribution.
\end{itemize}
\noindent\rule{16cm}{0.4pt}
%///////////////////////////////////////////////////////////////////////
\begin{problem} (\textit{Pricing exotic option})
\end{problem}
\begin{proof} Consider the replicating portfolio $X_t$ that contains $\Delta_t$ shares of stock and $(X_t - \Delta_t S_t)$ in bank account. We have:
\begin{equation}
	dX_t = \Delta_t dS_t + (X_t - \Delta_t S_t)rdt
\end{equation}
Since $X_t$ is the replicating portfolio, $X_t$ equals the non-arbitrage price of the option: $X(t) = c(t,S_t)$. Apply Ito's lemma on option price, and use the fact that $d[S,S](t)=\sigma^2S_t^2dt$
$$
dc(t,S_t) = \partial_t c_tdt + \partial_x c_tdS_t + \frac{1}{2}\partial_{xx}c_t \sigma^2S_t^2 dt
$$
We choose $\Delta_t = \partial_x c_t$, $X(0) = c(0,S(0))$ $\Rightarrow$
\begin{equation}
	\begin{split}
		&\partial_t c_tdt + \partial_x c_tdS_t + \frac{1}{2}\partial_{xx}c_t \sigma^2S_t^2 dt = \partial_xc_t dS_t + (c_t - \partial_xc_t S_t)rdt \\
		\Rightarrow ~~~&\partial_t c_t + rS_t\partial_x c_t + \frac{1}{2}\partial_{xx}c_t \sigma^2 S_t^2 - rc_t = 0
	\end{split}
\end{equation}
Consequently, $c(t,x)$ solves the boundary value problem:
\begin{equation}
	\begin{cases}
	\partial_t c + rx\partial_x c + \frac{1}{2}\sigma^2 x^2\partial_{xx}c  - rc = 0 & 0\leq t < T\\
	c(t,0) = 0 & t\leq T \\
	c(T,x) = x^{\gamma} &x > 0
	\end{cases}
\end{equation}
As the hint suggests, we (magically) know that the solution takes a separated form: $c(t,x) = f(t)g(x)$. \\
Plug in the boundary conditions:
\begin{equation}
	c(T,x) = f(T) g(x) = x^{\gamma}
\end{equation}
Hence we deduce that $g(x) = C_1 x^{\gamma}$, $C_1$ is constant. Look back into the PDE, with $\partial_t c = f'g, \partial_x c = fg', \partial_{xx}c = fg''$:
\begin{equation}
	\begin{split}
		&f'g + rx fg' + \frac{1}{2}\sigma^2 x^2 fg'' - rfg = 0 \\
		\Rightarrow ~~~~&\frac{f'(t)}{f(t)} = \frac{-rxg'(x) - \frac{1}{2}\sigma^2 x^2 g''(x) + rg(x)}{g(x)}
	\end{split}
\end{equation}
By the separation of variable method, there exists a constant $\lambda$, such that
\begin{equation}
	\begin{cases}
	f'(t) = \lambda f(t) \\
	-rxg'(x) - \frac{1}{2}\sigma^2 x^2 g''(x) + rg(x) = \lambda g(x)
	\end{cases}
\end{equation}
Plug in $g(x)=C_1x^{\gamma}$ to the second equation:
\begin{equation}
	\begin{split}
		&-r \gamma C_1 x^\gamma - \frac{1}{2}\sigma^2 \gamma(\gamma-1)C_1x^{\gamma} + rC_1 x^{\gamma} = \lambda C_1 x^{\gamma} \\
		\Rightarrow~~~& \lambda = -\left(r(\gamma-1) + \frac{1}{2}\sigma^2\gamma(\gamma-1)\right)
	\end{split}
\end{equation}
Plug into the first equation:
\begin{equation}
	f(t) = f(0) e^{\lambda t} = f(0)e^{-\left(r(\gamma-1) + \frac{1}{2}\sigma^2\gamma(\gamma-1)\right)t}
\end{equation}
Hence,
\begin{equation}
	\begin{split}
		&c(T,x) = f(T)g(x) =f(0)e^{-\left(r(\gamma-1) + \frac{1}{2}\sigma^2\gamma(\gamma-1)\right)T}C_1 x^{\gamma} = x^{\gamma} \\
		\Rightarrow ~~~& f(0)C_1 = e^{\left(r(\gamma-1) + \frac{1}{2}\sigma^2\gamma(\gamma-1)\right)T}
	\end{split}
\end{equation}
Therefore
\begin{equation}
	\begin{split}
		c(t,x) &= f(t)g(x) = f(0)C_1e^{-\left(r(\gamma-1) + \frac{1}{2}\sigma^2\gamma(\gamma-1)\right)t} x^{\gamma} \\
		&= e^{\left(r(\gamma-1) + \frac{1}{2}\sigma^2\gamma(\gamma-1)\right)(T-t)} x^{\gamma}
	\end{split}
\end{equation}
\end{proof}

\noindent\rule{16cm}{0.4pt}
%///////////////////////////////////////////////////////////////////////
\begin{problem} (\textit{Volatility arbitrage})
\end{problem}
\begin{proof} \textbf{(a)} We first formalize what $\sigma_1$ and $\sigma_2$ really is, because this will affect our final conclusion significantly.
\begin{itemize}
	\item[1.] As the problem itself suggests, $d_{\pm} = d_{\pm}(T-t, x; \sigma_1)$ is parametrized by $\sigma_1$, hence $c = c(t,x;\sigma_1)$. Therefore, $\sigma_1$ is the volatility that the market uses to price the option, i.e. the implied volatility.
	\item[2.] Our stock position is given by $\Delta = N(d_+)$, note that $d_+ = d_{\pm}(T-t, x; \sigma_1)$, so $\Delta = \Delta(t, S_t; \sigma_1)$. We are actually delta hedging with the \emph{implied volatility}. This is an important assumption, if we hedge with some other volatility parameter, the final result can be very different. See Carr et.al[2005]. 
	\item[3.] Finally, $dS_t = \alpha S_t + \sigma_2 S_t dW_t$, which implies that $\sigma_2$ is the real diffusion parameter underlying the data generating process of stock price.
\end{itemize}
As the hint suggests,
\begin{equation}
	dX_t = dc_t - \partial_x c_t dS_t + (X_t - c_t + S_t \partial_xc_t)rdt
\end{equation}
Therefore
\begin{equation}
	\begin{split}
		d(e^{-rt}X_t) &= -r e^{-rt}X_tdt + e^{-rt}dX_t \\
		&= -r e^{-rt}X_tdt + e^{-rt}\left(dc_t - \partial_x c_t dS_t + (X_t - c_t + S_t \partial_xc_t)rdt\right) \\
		&= e^{-rt}\left(dc_t - \partial_x c_t dS_t + (- c_t + S_t \partial_xc_t)rdt\right) ~~~(\dag)
	\end{split}
\end{equation}
The quadratic variation of $S_t$ is $d[S,S](t) = \sigma_2^2 S_t^2 dt$. Apply Ito's lemma for $c_t$:
\begin{equation}
	\begin{split}
		dc_t(t,S_t) = \partial_t c_t dt + \partial_x c_t dS_t + \frac{1}{2}\partial_{xx}c_t \sigma_2^2 S_t^2 dt
	\end{split}
\end{equation}
Plug into $(\dag)$:
\begin{equation}
	\begin{split}
		d(e^{-rt}X_t) &= e^{-rt}\left(\partial_t c_t dt + \textcolor{red}{\partial_x c_t dS_t} + \frac{1}{2}\partial_{xx}c_t \sigma_2^2 S_t^2 dt - \textcolor{red}{\partial_x c_t dS_t} + (- c_t + S_t \partial_xc_t)rdt\right) \\
		&= e^{-rt}\left[\underbrace{\left(\partial_t c_t dt + rS_t \partial_xc_t -rc_t\right)}_{\text{Use BS PDE parametrized by~}\sigma_1}dt + \frac{1}{2}\partial_{xx}c_t \sigma_2^2 S_t^2 dt\right]\\
		&= e^{-rt}\left[-\frac{1}{2}\partial_{xx}c_t \textcolor{red}{\sigma_1^2} S_t^2 dt+ \frac{1}{2}\partial_{xx}c_t \textcolor{blue}{\sigma_2^2} S_t^2 dt\right]\\
		&= \frac{1}{2}e^{-rt}\partial_{xx}c_t (\sigma_2^2-\sigma_1^2) S_t^2 dt
	\end{split}
\end{equation}
Integrate both sides from $0$ to $T$:
\begin{equation}
	\begin{split}
		e^{-rT}X(T) - 0&= \int_0^T\frac{1}{2} e^{-rt}\partial_{xx}c_t (\sigma_2^2-\sigma_1^2) S_t^2 dt \\
		\Rightarrow~~~X(T) &=\int_0^T \frac{1}{2} e^{r(T-t)}\partial_{xx}c_t (\sigma_2^2-\sigma_1^2) S_t^2 dt
	\end{split}
\end{equation}
Since $\sigma_1$, $\sigma_2$ are constants, everything else inside the integral is positive. Hence we have:
$$
\text{sgn}(X(T)) = \text{sgn}(\sigma_2 - \sigma_1)
$$
~\\
\textbf{(b)} In this case
\begin{equation}
	X(T) =\int_0^T \frac{1}{2} e^{r(T-t)}\partial_{xx}c_t (\sigma_2^2(t)-\sigma_1^2) S_t^2 dt
\end{equation}
The quantity $\sigma_2^2(t)-\sigma_1^2$ is, however, random. Thus we can not say about the sign of $X(T)$ this time.
\end{proof}

\noindent\rule{16cm}{0.4pt}
%///////////////////////////////////////////////////////////////////////
\begin{problem} (\textit{Asian option})
\end{problem}
\begin{proof} \textbf{(a)} By the hint:
\begin{equation}
	dX_t = \Delta_t dS_t + (X_t - \Delta_t S_t) rdt
\end{equation}
By the multi-dimensional Ito's lemma: (with shorthand notation $f_t:=f(t,S_t,Y_t)$)
\begin{equation}
	\begin{split}
		d(f(t,X_t,Y_t)) &= \partial_t f_t dt + \partial_x f_t dX_t + \partial_y f_t dY_t \\
		&~~~+\frac{1}{2}\left(\partial_{xx}f_td[X,X](t) + 2\partial_{xy} f_t d[X,Y](t)+\partial_{yy}f_td[Y,Y](t)\right)
	\end{split}
\end{equation}
Here we have $X_t = S_t$, $Y_t = \int_0^t S(s)ds$ is of bounded variation. Hence $d[Y,Y](t)=0$, $d[X,Y](t)=0$. Moreover, $d[S,S](t)=\sigma^2 S^2dt$, $dY_t=S_tdt$ The formula of $df$ reduces to
\begin{equation}
	\begin{split}
		d(f(t,S_t,Y_t)) &= \partial_t f_t dt + \partial_x f_t dS_t + S_t\partial_y f_t dt + \frac{1}{2}\sigma^2 S_t^2 \partial_{xx}f_tdt \\
		&=\left(\partial_t f_t+S_t\partial_y f_t+\frac{1}{2}\sigma^2 S_t^2 \partial_{xx}f_t\right)dt + \partial_x f_t dS_t
	\end{split}
\end{equation}
Since $f(t,S_t,Y_t)=X_t \Rightarrow$ $d(f(t,S_t,Y_t)) = dX_t$, hence
\begin{equation}
	\begin{cases}
	\partial_x f_t = \Delta_t \\
	\partial_t f_t+S_t\partial_y f_t+\frac{1}{2}\sigma^2 S_t^2 \partial_{xx}f_t = r(X_t - \Delta_t S_t) = rf_t -rS_t\partial_x f_t
	\end{cases}
\end{equation}
Hence $f(t,x,y)$ satisfies PDE:
\begin{equation}
	\partial_t f +rx\partial_x f  + \textcolor{red}{x\partial_y f}+\frac{1}{2}\sigma^2 x^2 \partial_{xx}f -  rf = 0 ~~~~~(\ddag)
\end{equation}
~\\
\textbf{(b)} Since $c(t,x,y)$ is a function such that $c(t,S_t,Y_t)$ is AFP $\Rightarrow$ $c(t,S_t,Y_t)=X_t$. So $c(t,x,y)$ also solves $(\ddag)$.  \\
We consider the boundary conditions.
\begin{itemize}
	\item[1.] When $t=T$, clearly by the option contractual terms: $c(T,x,y) = (y/T-K)^{+}$.
	\item[2.] When $x=0$ at somewhere \emph{before} the maturity, i.e. $0\leq t<T$: The stock price will remain 0 from that point, otherwise there is an arbitrage. So 
	$$
	Y(T) = \int_0^T S(s)ds = \int_0^t S(s)ds + \int_t^T 0 \cdot ds = Y(t) = y
	$$
	i.e. we know as early as time $t$ that the path average at expiry will be $y/T$, and the payoff will be $(y/T - K)^+$. This turns our option into a deterministic cashflow that is like a zero coupon bond. So
	$$
	c(t,0,y) = e^{-r(T-t)}(y/T - K)^+
	$$
	\item[3.] If $y\to -\infty$, the value of option will be simply $0$.
\end{itemize}
To wrap up: $c(t,x,y)$ solves the BVP:
\begin{equation}
	\begin{cases}
	\partial_t c +rx\partial_x c  + x\partial_y c+\frac{1}{2}\sigma^2 x^2 \partial_{xx}c -  rc = 0 & t\in[0,T), x\geq 0\\
	c(T,x,y) = (y/T-K)^{+} & x\geq0\\
	c(t,0,y) = e^{-r(T-t)}(y/T - K)^+ & t\in[0,T)\\
	\lim\limits_{y\rightarrow -\infty}c(t,x,y) = 0 & t\in[0,T), x\geq 0
	\end{cases}
\end{equation}
~\\
\textbf{(c)}. We structure portfolio $X(t)$, with $\Delta_t = \partial_x c$ shares of stock and $X(t)-\Delta_t S_t$ dollar in bank account. Pick the initial value of the bank account such that $X_0=c(0,S_0,Y_0)$, which is a known constant. It suffices to show two things:
\begin{itemize}
	\item[1.] $X_t$ is the replicating portfolio $\iff$ $X_T = (Y_T-K)^+$.
	\item[2.] $X_t = c(t,S_t,Y_t)$. For $0\leq t \leq T$.
\end{itemize}
We begin with (2), let $D_t = X_t - c(t,S_t,Y_t)$, then by our results in \textbf{(a)}:
\begin{equation}
	\begin{split}
		dD_t &= dX_t - dc(t,S_t,Y_t) \\
		&=\left[\Delta_t dS_t + (X_t - \Delta_t S_t) rdt\right] - \left[\left(\partial_t c+S_t\partial_y c+\frac{1}{2}\sigma^2 S_t^2 \partial_{xx}c\right)dt + \partial_x c dS_t\right] \\
		&=-\underbrace{\left(\partial_t c+ rS_t\partial_x c + S_t\partial_y c+\frac{1}{2}\sigma^2 S_t^2 \partial_{xx}c\right)}_{\text{Use the PDE satisfied by~}c}dt + rX_t dt \\
		&= r(X_t - c(t,S_t,Y_t)) dt \\
		&= D_t rdt
	\end{split}
\end{equation}
So $D(t) = D(0) e^{rt}$, since $D(0)=0 \Rightarrow D(t)\equiv 0$ for $t \in [0,T)$. Since $X,c,D\in \mathcal{C}^{1,2,2} \Rightarrow X,c,D$ are all continuous with respect to $t$: $D(T)=0$, i.e. $X(T)=c(T,S_T,Y_T)=(Y_T/T-K)^+$. We proved condition (1).\\
Besides, $D(t)=0$ for all $t\in[0,T] \Rightarrow $ $X_t = c(t,S_t,Y_t)$ for all $t\in[t,T]$. We have condition (2).\\
$\Rightarrow c(t,S_t,Y_t)$ is the AFP.

\end{proof}






\end{document}